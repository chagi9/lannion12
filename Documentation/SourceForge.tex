\documentclass[a4paper,12pt]{article}

\usepackage[utf8]{inputenc}
\usepackage[T1]{fontenc}
\usepackage[normalem]{ulem}
\usepackage[french]{babel}
\usepackage{verbatim}
\usepackage{graphicx}
\usepackage{anysize}

\marginsize{15mm}{14mm}{12mm}{10mm}

\title{Analyse du site SourceForge}
\author{Arthur Crenn et Charlotte Girouard}

\addtolength{\topmargin}{-50 pt}
\addtolength{\evensidemargin}{-30 pt}
\addtolength{\oddsidemargin}{-13 pt}
\addtolength{\textwidth}{40 pt}
\addtolength{\textheight}{50 pt}
\addtolength{\footskip}{12 pt}

\begin{document}

\maketitle

\section{SourceForge}

\vskip5mm

\subsection{Projets}

\begin{itemize}

\item Recherche d'un projet parmi tous ceux existant.
\item Description complète du projet (Auteur, fonctionnalités, SVN etc ...).
\item Parcours de tous les fichiers d'un projet et possibilité d'en télécharger un ou plusieurs.
\item Nombreuses statistiques ( nombre de téléchargements, nombre de bugs etc ... ).
\item Présence de tris ( dernier publié, les plus populaires etc ... ).
\item Projets recensés en différentes catégories.
\item Commentaires, notes et bugs d'autres utilisateurs du projet.
\item Forum d'aide.

\end{itemize}

\vskip5mm

\subsection{Utilisateur}

\begin{itemize}

\item Espace de connexion et de création d'un nouveau compte.
\item Possibilité de publier, de rejoindre ou de créer un projet une fois connecté.
\item Répondre à un sujet sur le forum, une fois connecté.

\end{itemize}

\vskip5mm

\section{Notre site}
Après avoir analyser le site nous avons décidé de prendre les fonctionnalités suivantes :

\subsection{Projets}

\begin{itemize}

\item Parcours de tous les fichiers d'un projet et possibilité d'en télécharger un ou plusieurs.
\item Description complète du projet (Auteur, fonctionnalités, SVN etc ...)
\item Rapports de bugs des utilisateurs sous forme de commentaires.
\item Statistiques ( Compteur de téléchargement etc ...) 

\end{itemize}

\subsection{Utilisateur}

Chaque utilisateur aura accès aux fonctionnalités décrites ci-dessus.

\vskip5mm

\begin{itemize}

\item Visiteur :

\vskip3mm

\subitem\textbullet{} Enregistrement d'un nouveau compte en tant que contributeur.

\vskip3mm

\item Contributeur :

\vskip3mm

\subitem\textbullet{} Permet d'envoyer des modifications sur un projet.

\end{itemize}

\end{document}